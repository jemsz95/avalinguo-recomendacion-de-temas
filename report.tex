\documentclass[journal]{IEEEtran}

\hyphenation{op-tical net-works semi-conduc-tor}

\usepackage{graphicx}
\usepackage{mathtools}

\begin{document}

\title{
Combining extrinsic and intrinsic recommendation systems with
absolute popularity metrics for producing a better topic suggestion: to
groups of individuals with related interests
}

\author{
Javier~Meza,~\IEEEmembership{ITESM},
Daniel~Hernández,~\IEEEmembership{ITESM}

\thanks{M. Shell was with the Department
of Electrical and Computer Engineering, Georgia Institute of Technology,
Atlanta, GA, 30332 USA e-mail:
(see http://www.michaelshell.org/contact.html).}
}

% The paper headers
\markboth{Proyecto de sistemas inteligentes II}{
Meza \MakeLowercase{\textit{et al.}}: Combining extrinsic and intrinsic
recommendation systems with absolute popularity metrics for producing a
better topic suggestions to groups of individuals with related interests
}

\maketitle

\begin{abstract}
For a couple of years now, recommendation systems have been combined to
complement each others’ bias and produce better predictions. The typical
approach includes combining collaborative filtering with other intrinsic
recommendation systems. These combined systems are proven to
yield positive results and will be explored upon in this paper as basis
for a new group recommendation solution.
\end{abstract}

\begin{IEEEkeywords}
intelligent systems, machine learning, recommendation.
\end{IEEEkeywords}

\IEEEpeerreviewmaketitle


\section{Introduction}

\IEEEPARstart{A}{valinguo} is a research project and start-up with a
very high ambition: revolutionizing the way people learn languages using
Virtual Reality and Artificial Intelligence. Supported by the claim that
languages need to be practiced by speaking and not by memorizing or
writing, Avalinguo connects people from all around the world in virtual
rooms where language learners can discuss topics of their interest and
thus practice and improve their skills at any given language. The
technology behind this is promising, to say the least: Machine Learning
algorithms have already been applied to determine the fluency skills of
non-native English speakers, and there are other challenging but
rewarding projects to be completed. In the following pages, we present
the report of the completion of one such project.

One key aspect that ensures user retention within Avalinguo is the
quality of the interactions in each virtual room. Among the factors that
influence this quality lies the selected topic for the virtual room. An
optimal selected topic is one that considers the interests of every 
user, the popularity of a given set of topics and the pool of current
trending topics. Then, from all the users within the app that want to
learn the same language and are at approximately the same level, the
ones that have an overall similar vector of interests should be chosen
and a topic that is in accordance with those interests will be then 
selected. However, this reasoning had not been applied to select the
users to participate in each room, so that is what this report is 
focused on: the process we went through to consider all the interests 
and preferences of all users within the platform so that the users in
each virtual room are as similar in interests as possible, therefore
making their interactions more meaningful and increasing the chances of
their return to the platform.


\section{Previous work}

Creating groups out of individuals of similar interests was a research
topic carried out by Miguel Bravo for the Avalinguo project. Part of
Bravo`s research revolved around creating a synthetic data-set of users
and their interests based on the interest distribution of users. Bravo
obtained this distribution information from Facebook`s advertising
solution. Using this synthetic data-set proved to be key to evaluating
our recommendation system.

Our solution, though focused in a new area, continues on with a similar
problematic. Using each homogeneous group as a singular unit we may
recommend conversation topics with high confidence due to the low
variance in user interest.


\section{Recommending to individuals}

As a first step to the creation of a verifiable recommendation system we
went about collecting topics of interest that came from human generated
sources. In order to keep them up-to-date, we used data from Google
Trends and the Twitter Trending API.

The second step was labeling these topics with meta data information. We
were able to get about five hundred trending topics and labeled each 
according to their affinity to the interest categories Bravo had laid
out for us. To facilitate this work, we created a web tool for manually
categorizing each topic, assigning it the labels we deemed appropriate.

\subsection{Categorization}

Bravo's categorization solution has a tree structure, in which each
level gives a more specific category description of its parent as shown
in figure \ref{fig:interest_tree}. We parsed each user`s interest tree
into a weighted vector according to the proportion each label
contributes to its parents' weights. Each interest weight \(W_i\) with
\(C\) children and \(S\) siblings is calculated as follows:

\[
    W_i = 
    \begin{dcases}
        \frac{1}{S},
                & \text{if tagged}\\
        \frac{1}{S} \sum_{k}^{\text{children}}\frac{W_k}{C},
                & \text{otherwise}
    \end{dcases}
\]

In the same fashion, the interest vector of each topic was calculated. 
From here a distance metric can be established to represent the affinity
of each user to all topics. In the case of this project, the euclidean
distance was selected.

\subsection{Collaborative filtering}

Our user selection is based on multiple factors, the first of which
consists of extrinsic recommendations. That is, users who have
similarities when it comes to their preferences about a certain topic.
To calculate this, we obtained the correlation of each user’s vector of
interests by computing the sum of each vector’s distance from the
average, and then normalized it by dividing it by the Euclidean
distance. We used Pearson`s collaborative filtering formula to calculate
how similar topics were among a group of users with similar interests. 
Each topic`s similarity was calculated in the following fashion: 

\[
simil(x,y)=
\frac{\sum_{i}^{\text{users}} ({r}_{x,i} - \overline{r}_{x}) 
({r}_{y,i} - \overline{r}_{y})
}
{\sqrt{ \sum_{i}^{\text{users}} ({r}_{x, i} - \overline{r}_{x} )^2 }
\sqrt{ \sum_{i}^{\text{users}} ({r}_{y, i} - \overline{r}_{y} )^2 }
}
\]

where ${r}_{x,i}$ represents each user`s x-distance from the interest
mean and ${r}_{y,i}$ represents its y-distance from the interests mean.

\subsection{Google Ranks}

One of the challenges faced regarding the trending topics, was that the
trendiness of a topic  may not be enough when checking for the relevance
of the topic as a whole. That is why, as a measure of absolute
popularity for a given topic, we chose to obtain the number of search
results for every given topic, therefore obtaining more information on
how relevant that topic had been in longer periods of time

An equally important part of our work was obtaining a metric that
complemented the trendiness of a topic since topics may lose trendiness
over time and the goal of Avalinguo`s recommendation system is to
produce topic recommendations that keep relevant over time. In order to
accomplish this, we decided to take into account the number of times a
given topic had been searched on Google Search. We took each retrieved
topic and made a request to Google Search, which returned the HTML from
which we parsed the number of search results. These results were stored
in a file and were then normalized by applying the ${log}_{2}$ function.

\begin{figure}
    \centering
    \includegraphics[width=\linewidth]{InterestTree}
    \caption{Interest categorization tree}
    \label{fig:interest_tree}
\end{figure}

\section{Recommending to groups}

\appendices

\section{Proof of the First Zonklar Equation}
Appendix one text goes here.

% you can choose not to have a title for an appendix
% if you want by leaving the argument blank
\section{}
Appendix two text goes here.


% use section* for acknowledgment
\section*{Acknowledgment}
The authors would like to thank Dr. Ramon Brena for his invaluable
support in the development of this project. Each session of bouncing
ideas off one another resulted in more robust solution for Avalinguo. 

\begin{thebibliography}{1}

\bibitem{IEEEhowto:kopka}
H.~Kopka and P.~W. Daly, \emph{A Guide to \LaTeX}, 3rd~ed.\hskip 1em
  plus 0.5em minus 0.4em\relax Harlow, England: Addison-Wesley, 1999.

\end{thebibliography}

\end{document}
